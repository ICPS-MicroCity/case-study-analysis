\section{Ubiquitous Language}
\label{sec:ubiquitous-language}

The following table shows the ubiquitous language of the analyzed case study.

\begin{table}[H]
    \centering
    \begin{tabular}{|l|p{.7\textwidth}|}
        \hline
        \textbf{Term} & \textbf{Micro City's Term} & \textbf{Definition}\\
        \hline
        \ul{Amusement Park}{Micro City}{A large outdoor area with fairground rides, shows, and other entertainments.}
        \ul{Visitor}{Guest}{A person attending the amusement park.}
        \ul{Group of Visitors}{Group of Guests}{A set of visitors attending the amusement park.}
        \ul{Attraction}{Service}{Type of activity offered to visitors. An attraction is continuously available during the amusement park's lifetime and allows visitors to benefit from it at any time. They can be rides, roller coasters, water slides, but also restaurants.}
        \ul{Show}{Event}{Type of activity offered to visitors. A show takes place in a specific moment and is carried out only once; when it terminates, it won't be available anymore.}
        \ul{Satisfy}{Satisfy}{The action of an attraction or a show of providing visitors with an experience or a product.}
        \ul{Benefit From/Attend}{Benefit From/Attend}{The act of a visitor of exploiting an attraction or a show and being satisfied by it.}
        \ul{Time Period}{Time Period}{An attraction's operation time span. It is defined by a start and a finish.}
        \ul{Duration}{Duration}{The time taken by an attraction or a show to satisfy one or more visitors.}
        \ul{Waiting Time}{Waiting Time}{Amount of time that visitors wait before benefiting from an activity.}
        \ul{Queue}{Queue}{Set of aligned visitors due to long waiting time.}
        \ul{Wearable}{Wearable}{Device owned by each visitor (or group of visitors) that allows them to interact with the amusement park.}
        \ul{Recommendation}{Recommendation}{A proposal to benefit from a specific attraction or show in exchange for a reward.}
        \ul{Recommend}{Recommend}{The action of sending a recommendation to the visitors.}
        \ul{Accept a Recommendation}{The action of accepting a recommendation and performing the action recommended in order to receive a reward.}
        \ul{Reward}{The recompense received by the guests that accept a recommendation. It may be given by activities or the \textit{Micro City} itself in order to promote specific behaviours.}
    \end{tabular}
    \caption{Ubiquitous language of the amusement park's case study.}
    \label{tab:ul}
\end{table}
