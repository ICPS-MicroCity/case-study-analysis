\section{The Case Study}
\label{sec:case}

As mentioned in the introduction, the case study of the \textit{Micro City} analyzed in this document is the one of amusement parks.
Amusement parks are considered to be a large business all around the world and attract people of every age and culture.
They satisfy the \textit{Micro City}'s spatial and temporal characteristics as most of them develop in a bounded spatial area and are open daily for a limited amount of hours.
Usually, they are not as wide as a city and grant access to visitors in the daylight hours.

Inside, they offer a large set of attractions that may vary depending on the type of amusement park.
For instance, they may be roller-coasters, carousels, water slides and many others.
Attractions correspond to \textit{services} in the current case study, as they offer experiences to \textit{guests}.
On the other hand, shows correspond to \textit{events}, as they take place at a specific time.
Usually, attractions are static, that is, they do not move.
Moreover, they satisfy a certain amount of guests within a limited duration: this time span is identified by the attraction's frequency, that is the rate at which the attraction starts a ride.
In most of the cases, attractions have one or many workers that decide when a ride can start depending on a set of environmental factors, different for every type of attraction.
The number of guests that are satisfied in a ride corresponds to the number of people that fit in the attraction.

Guests are embodied by visitors.
They are highly interested in the activities offered by the amusement park.
In fact, they attend the parks mainly for the attractions inside them.
They may attend the parks as single individuals (for instance, buying a personal ticket) or in groups.
It can be assumed that they own a personal wearable device used to receive information about attractions.
Also, it can be assumed that a group of guests uses a single wearable device for the same purpose.
An amusement park also presents internal operators, distinguished from guests, that do not benefit from attractions;
instead, they manage the attractions and help visitors.

An amusement park may have its personal business model and could require to pay a fee just to get inside the park itself.
Attractions inside the park can be free or could need a form of payment from visitors because they offer additional services or products, such as restaurants.

In this scenario, it could be useful to recommend the most suitable attraction to visitors depending on their physical location, tracked by their personal wearable device, or on their interests.
The recommendations may concern the nearest attraction that suits the visitor's preferences or an attraction with a short queue (compared to the average queue of the attraction, or compared to the queues of other attractions).
This mechanism could be referred as \textbf{situated recommendation}.
By accepting situated recommendations, visitors may receive a reward provided by the amusement park.
The most straightforward reward is, for instance, the reduction of the waiting time in order to benefit from an attraction.
