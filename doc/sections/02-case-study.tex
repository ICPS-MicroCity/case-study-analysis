\section{The Case Study}
\label{sec:case}

As mentioned in the introduction, the case study of the \textit{MicroCity} analyzed in this paper is that one of amusement parks. Amusement parks are considered to be a large business all around the world and attract people of any age and culture. They satisfy the \textit{MicroCity}'s spatial and temporal characteristics as most of them cover a buonded spatial area and are open daily for a limited amount of hours. Usually, they are not as wide as a city and grant access to visitors in the daylight hours.

Inside, they offer a large set of attractions that may vary depending on the type of amusement park; for instance, they may be rollercoasters, carousels, water slides and many others, including events such as shows. Attractions correspond to activities in the current case study, as they are services that offer experiences to guests. Usually, attractions are static, that is they do not move. At last, they satisfy a certain amount of guests within a limited duration: this time span is identified by the attraction's frequency, that is the rate at which the attraction starts a round. In most of the cases, attractions have one or many workers that decide when a round can start depending on a set of environmental factors, different for every type of attraction. The number of guests that are satisfied in a round corresponds to the number of people that fit on the attraction.

Guests are embodied by visitors. They are highly interested in the activities offered by the amusement park; moreover, they attend the parks mainly for the attractions inside them. They may attend the parks as single individuals (for instance, buying a personal ticket) or in groups. Nowadays, every person that attends an amusement park owns a smartphone; so, it is assumed that they can use it as the personal wearable device used to receive information about attractions. Also, is assumed that a group of guests uses a single wearable device in order to benefit from attractions. An amusement park also presents internal operators, distinguished from guests, that do not benefit from attractions; instead, they manage them by reducing the queue and helping visitors.

An amusement park may have its personal business model and could make it necessary to pay a fee just to enter into the park. Attractions inside the park can be free or could need a form of payment for visitors because they offer additional services or products, such as restaurants.

In this scenario, it could be useful to recommend the most suitable attraction to visitors depending on their physical location, tracked by their personal wearable device. The recommendation may concern the nearest attraction that suits the visitor's preferences or an attraction with a short queue (compared to the average queue of the attraction). By accepting recommendations, visitors may receive a reward provided by the amusement park.